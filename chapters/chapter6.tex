\chapter{Conclusion}
\markright{CHAPTER 6}
%
\section{Summary}
% one paragraph for the whole
In the context of promoting the use of public transit, the prediction of rail transit ridership is becoming more and more important. This research taking explaining the rail transit ridership as the overall goal estimated the influences of various factors from the perspectives of station level and station-to-station level respectively. Moreover, this research also provided new explanations for the catchment area of rail transit stations. As results, this research provided an approach to select the valid indicators; and proposed a ridership forecasting method considering the interactions among stations and stations; also, it showed a way to accurately estimate the catchment area.

% brief summary for each chapter
% Note: re-state the main points in a new concise way that you want your readers to remember.
Specific to each chapter, the main content and findings are:
\begin{itemize}
	\item \emph{\textbf{Chapter 1}} proposed the overall research purpose of exploring determinants of rail transit ridership based on the needs of sustainable urban development. By reviewing the literature relating to this field, specific research questions were proposed. Around the primary goal of exploring determinants of rail transit ridership, the description of study case and dissertation organization were given at the last of this chapter.
	
	
	\item \emph{\textbf{Chapter 2}} discussed how the walking duration to rail transit station is affected by passenger attributes. Centering with this topic, 3 specific research questions were proposed according to the previous studies, they are: 1. How to understand and describe the property of walking access to rail transit; 2. how to identify the valid factors influencing the preference to walking duration; 3. how to estimate the walking duration using passenger attributes. This chapter was organized centering the 3 research questions. In the beginning, a detailed interpretation of the property and implication of walking access to transit station was given, based on which the description model of the relationship between walking duration and passenger attributes was constructed. This study argued that the probability of walking a given walking duration or more should be influenced by passenger attributes, and converted this issue into a binary choice problem. And then, the ANOVA was used to identify the feature attributes of passengers at each given walking duration threshold. With the extracted feature attributes, the random decision forest model was adopted to explore preferences to walking duration of passengers with different attributes. The probabilities of walking more than the given thresholds of walking duration were estimated using the training set, and predicted with the test set. At last, the evaluation of the prediction showed that the individual behavior of walking more than a given walking duration still cannot be predicted accurately, but the overall tendency to walking duration of a group of passengers is predictable at some extent. As the conclusion, the quantitative relationship between walking duration and passenger attributes discussed in extensive literature was verified in this study, also the possibility of predicting walking duration was provided.
	
	\item \emph{\textbf{Chapter 3}} is a preliminary study of exploring the determinants of rail transit ridership. This chapter	summarized the characteristics of rail transit ridership and land-use in the case of Fukuoka. It aimed to make a comprehensive understanding of the research object of this dissertation, thus providing reference and implication for the next research. Based on this aim, chapter 3 focused on three aspects: 1. Summarizing the characteristics of rail transit ridership of Fukuoka; 2. Summarizing the characteristics of land-use around the rail transit stations in Fukuoka; 3. Exploring the relationship between rail transit ridership and the land-use around stations. Firstly, the characteristics of transit ridership were summarized from the perspectives of total amount, growth rate and spatial distribution. Then, the internal relationships between each type of land-use around the stations were interpreted using correlation analysis and further analyzed using factor analysis, based on which the subway stations were classified into 5 types in terms of the characteristics of land-use. At last, the influence of land-use around the stations on both the amount and growth rates of transit ridership was estimated using the quantification method \uppercase\expandafter{\romannumeral1}. To explore and estimate the influencing factors of transit ridership, some recommendations obtained from the result of this chapter: 1. pedestrian area but not radius buffer should be considered; 2. the explanatory variables should be enriched; 3. the estimation model should be improved; 4. the approach to addressing small sample case should be considered.
	
	\item \emph{\textbf{Chapter 4}} explored and estimated the influencing factors of rail transit ridership at the station level using the case of Fukuoka which has a small sample size. With this small sample study case, this chapter focused on 3 specific research contents: 1. Summarizing the literature and putting forwards the candidate indicators that may have the impact on the rail transit ridership; 2. improving the approach to identifying and selecting the valid indicators towards the case having a small sample size; 3. improving the estimation of Mix Geographically Weighted Regression by distinguishing the local/global variables.	This chapter was organized centering the 3 contents. First, the indicator system that is considered to have an impact on rail transit ridership was constructed from three categories of built environment, traffic accessibility, and social demographic environment. Particularly, the impact of the bus system was considered to have both positive and negative effect, represented by bus accessibility and bus capacity respectively. And then, to reduce the probability of type \uppercase\expandafter{\romannumeral1} and type \uppercase\expandafter{\romannumeral2} statistical errors in a small sample case, the exploratory regression was introduced to help to identify the valid explanatory indicators among the candidate indicators. Finally, the influence of the identified indicators was estimated using MGWR, where the local and global explanatory variables in MGWR were distinguished based on the spatial autocorrelation. As the conclusion, the approach proposed in this chapter is verified to be effective against identifying valid explanatory indicators in terms of small sample cases; the impact of the bus system was verified that it has both positive and negative effects on the rail transit ridership.
	
	\item \emph{\textbf{Chapter 5}} explained the how land-use patterns influence transit ridership at station-to-station level. The transit ridership at station-to-station level is a result of passenger transfer from station to station. With the main purpose of describing and estimating this passenger transfer, and the specific research contents were putting forward as followings. 1. Describing the passenger transfer from station to station, and convert it into a mathematical problem that can be estimated; 2. estimating the influence of land-use pattern of the catchment area on that passenger transfer. In this chapter, the passenger transfer was described using the probability of alighting at a specific transit station. This choice of destination was thought to be affected by the land-use type around the destination station. Therefore, this issue could be converted into a binary choice problem that the choice of whether alighting at a specific transit station is affected by the land-use type around that transit station. Then this binary choice problem was estimated using the logistic regression model. The results showed that the land-use type around a station has a significant influence on choosing that station as the destination. As the conclusions, 1. the probability of choosing a station as the destination tends to decrease if the land-use types are similar between the departure and destination stations; 2. the probability of choosing a station as the destination which belongs to low-density residence type has no tendency to raise regarding the variation in land-use in the departure station; 3. for any type of stations, the education land-use in the departure station contributes to an increase in the probability of choosing that station as the destination.
	
	\item \emph{\textbf{Chapter 6}} summarized the main content and findings from the view of both integer and each chapter. Recommendations of future work were also given for extending and improving this research field.
\end{itemize}

\section{Contributions}
% 贡献点,直接说本文主要的贡献点如下即可
% in bulleted list, format as below
% 做一个列表,加上label,在recommendations中进行引用
% format: problem description/contributions/chapter

% 将bus指标根据其对客流量的影响效果分为了bus capacity和bus accessibiliy,并验证了这两个指标的作用
% 引入exploratory regression进行筛选指标,并在小样本中证实了其有效性。
% 通过空间分布关系,确定MGWR中的local和global变量,the estimation of model 验证了有效性。
% 建立了乘客目的地选择的logistic regression模型来描述站与站之间的关系
% 定量的验证了土地利用类型对于乘客出行目的地选择的影响
% 对walking duration to transit station的含义重新进行了discuss
% 阐述了surveyed walking duration和walking preference的关系
% 通过random forest decision model从概率的角度对surveyed walking duration 和walking preference的关系进行了estimation
This dissertation worked on several key questions about the field of explaining rail transit ridership, the main contribution can be arranged as below.

\begin{enumerate}
	\item Reinterpreted the implication of being surveyed walking access to transit stations, that it has no linear relation to people willingness but just the reflection of the distance between departures and stations. (refer to \textbf{chapter 2})
	
	\item Described the correlation between surveyed walking duration and people's individual characteristics from the view of probability, which can be further applied to the estimation of catchment area of rail transit stations. (refer to \textbf{chapter 2})
	
	\item Analyzed the trend of variations in transit ridership, and classified the subway stations of Fukuoka into 5 types in terms of land-use. Confirmed the correlations between transit ridership and land-use. (refer to \textbf{chapter 3})
	
	\item Considered the influence of bus on rail transit ridership from both the sides of bus capacity and bus accessibility, and verified that the effect of bus capacity is positive to rail transit ridership, while the effect of bus accessibility is negative. (refer to \textbf{chapter 4})
	
	\item Proposed the approach to screen valid indicators by introducing the exploratory regression, and confirmed its effectiveness in a small sample case. (refer to \textbf{chapter 4})
	
	\item Distinguished the local and global variables in MGWR model by examining the spatial distribution of each variable, and the effectiveness was confirmed in the estimation of MGWR model. (refer to \textbf{chapter 4})
	
	\item By examining the probability of selecting the destination station from all the stations, established a logistic regression model for describing the correlation between stations and stations. (refer to \textbf{chapter 5})
	
	\item Quantitatively validated the impact on the selection of destination stations in terms of land-use types . (refer to \textbf{chapter 5})
	
\end{enumerate} 

\section{Recommendations}
% 对现状的建议
This research focused on the influencing factors of rail transit ridership from both station level and station-to-station level, also analyzed the key element influencing the use of rail transit, the scale of catchment area and walking preferences. Based on the findings in this research, several recommendations of helping increase the use of rail transit are outlined.

%
\begin{enumerate}
	%
	\item People's walking preferences have significant influence on the use of rail transit. It is suggested that the design for pedestrian accessibility should also consider the individual characteristics of resident, thus to help increase the use of rail transit.
	
	%
	\item With the development of rail transit, if viewing from the whole transit system, the positioning of bus system is gradually inclining to be the connection between departures and rail transit. Rather than planning higher capacity for bus transport, it is recommended planning more accessible bus routes which can help people use rail transit more convenient.
	
	% 功能较为集中的区域对于轨道交通的需求更大,在线路规划上建议将更多的功能区连接起来。
	\item According to the findings, the more aggregation in land-use functions the more demand on rail transit. It is recommended connecting more functional regions as possible when making rail transit planning.
\end{enumerate}

% 给未来的
At current stage, the obstacles of going deep into this research field are mainly in obtaining available data. Some suggestions for future work include the followings:

\begin{enumerate}
	\item Surveys of income and occupation, exploring the effect on travel behavior.
	
	\item Integer of the whole process of estimating the catchment area of rail transit stations based on the achievements of this dissertation.
	
	\item Joint of the estimation of catchment area and ridership forecasting, improving the accuracy.
	
	\item Analysis of the influence on land-use affected by the use of rail transit, exploring the interaction between land-use and transit.
	
	\item Analysis of balance condition among various elements in the catchment area of transit stations, including human, resource, land-use, transportation etc. achieving the goal of sustainable development.
\end{enumerate}

% reference
\clearpage % 新起一页
\bibliographystyle{apacite}
% \bibliographystyle{IEEEtran}
% \bibliography{ref}