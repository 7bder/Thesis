\chapter{Introduction}
\markright{CHAPTER 1}

%
\section{Background}

\subsection{The demand of sustainability caused by urbanization}
% status quo
With the continued urbanization, people keep on migrating from rural areas to urban areas. As a result of this, 54\% population of the world lives in urban areas by 2014, this population shift will continue and is predicted to reach 66\% by 2050 \cite{un2014world}. Living in cities is considered much more efficient and convenient than that in rural areas in a variety of ways, as it is easier to provide services when people live closer together. However, cities also change the way that people interact with each other and the environment, which often causes multiple problems. The second UN Conference on Human Settlements in 1996 came to the conclusion that the cities all over the world are facing problems due to urbanization.

% Different situations in developing countries and developed conutries
Although the problems commonly exist all over the world, the type and scale differ from different stages of urbanization. As to developing countries which are now experiencing a rapid change caused by the rapid migrants from rural areas to urban areas, the problems mainly include traffic congestion, disorganization, and pollution. Essentially, most of the problems are caused by the imbalance of demand and supply in land-use, resource, and infrastructure. How to satisfy this demand is an important issue that the governments in developing countries have to consider and address. Unlike developing countries, developed countries mainly have the problems of aging population and low birthrate. With a rapidly increasing proportion of aged people in the population, governments are forced to increase expenditure of social security, adding to that the low birthrate, further intensifies the shrinking of working-age population, thus increasing the fiscal burden of governments. How to reduce the public financial expenditure and improve the efficiency of social operation has become a problem that the governments in developed countries have to face. 

% Summary of status quo
In some ways, the problems of developing country and developed country are not exactly the same. But in essence, both of them can be viewed as the problem of sustainability. Governments in the developing countries mainly focus on trying to increase the supply of services to satisfy the increasing population, while those in the developed countries are busy tackling the economic slowdown which is caused by decentralization and changing working patterns. Under such a reality, the sustainability of urban development has become a severe topic, of which the environment-friendly and efficient travel mode is widely regarded as one of the most important issues.

\subsection{The role of public transit in sustainable development}
% Introduce public transit
Giving priority to public transit is one of the important ways to help create sustainable cities since its characteristics of large-capacity, high-speed, and low-emission. Increasing the share ratio of public transit has been widely accepted as the main measure for reducing per capita energy consumption and promoting sustainability. But the planning and construction of public transit are not as simple as building a station there and waiting for passengers come. Lots of factors can influence the use of public transit, especially for factors within the catchment area of transit stations. It is generally accepted that a development of land-use and facilities coordinated with the transit station plays the key role in attracting people to use the public transit. Several common elements of a coordinate development of land-use and facilities have been discussed in the previous studies \cite{boarnet1997story,bernick1997transit,megally2001california,cervero2004transit}.

\begin{itemize}
	\setlength{\parskip}{0\baselineskip} % 设置段间距
	\item Mixed-use development
	\item Rail transit stations as cores
	\item Compactness
	\item Pedestrian-friendly
	\setlength{\parskip}{0.7\baselineskip} % 设置段间距
\end{itemize}

% rough definition for TOD
The guidelines summarized above has received a lot of attention by the governments which are regarded as the means of mitigating most of the common urban problems, such as traffic congestion, air pollution, and incessant sprawl \cite{cervero2002transit}. In a common sense, an area taking the station as the core typically has a central transit stop (such as a train station, or light rail or bus stop) surrounded by a high-density mixed-use area, with lower-density areas spreading out from this center. The densest areas are normally located within a radius of 400 meters to 800 meters (5 minutes to 10 minutes walking duration, varying in terms of different kinds of transit station) around the central transit stop, which are determined by the acceptable walking distance of pedestrians, thus helping solve the last mile problem.

% Change the objective to rail transit station. Interpret the importance of ridership prediction
With the increase in demand for speed, punctuality and environment protection, urban rail transit has gained popularity by governments, especially in metropolises with high compactness and population density. Add to the continuous extension of city scale and the growth of travel demands, taking rail transit station as the core of a newly planned area is becoming more and more common. With this popularity, it is probably easy to enter a misunderstanding that if the transit station is built, people will come, but the reality is not that simple. Because of this, understanding the determinants of rail transit ridership has become central and fundamental to decisions on urban planning and management. 

% 
\section{Research purpose}
% propose station-level and station-to-station level
What explains rail transit ridership? How to increase the rail transit ridership? As interpreted before, these questions are placed in front of us. The answers seem to be both obvious and complex. The elements existing within the catchment area of stations should associate with rail transit ridership, like population, road network, parking, income, transit network, building density etc. But the relative importance of these various factors and the internal relationships among them are much more complex, and still not well understood \cite{taylor2003factors}. Besides, as transit station is a part of transit network but not an independent existence, once the transit ridership of one station varied, transit ridership of all the other stations connected to that station should vary as well. But the interaction among catchment areas of the stations in the same transit network remains unclear to date.

%  catchment area
We mentioned the term of catchment area above, however, what does it mean in transit ridership analysis? As a general definition, a catchment area is an area that the rail transit station can serve. In the issue of transit ridership analysis, it means a station's primary service area, within which people are willing to use the station, and beyond which land-use, travel mode choice, population distribution are unlikely to be influenced by this transit station. The catchment area is thought to be largely determined by people's willingness of walking to transit stations, which, however, vary by people's individual characteristics including trip purpose, age, gender etc. \cite{guerra2013half}. Because of this, the transit catchment area is important and fundamental to the ridership analysis, even the most.

% 基于上述存在的现实问题,我们得出了explain rail transit ridership 这个整体目标。以此为中心,本文针对以下三个问题点进行展开
Based on the interpretation of the real issue above, the research goal of the whole dissertation is defined to \emph{\textbf{explore the determinants on the utilization of rail transit}}. Centering on this theme, three subtopics are proposed to be discussed in this dissertation.

\begin{enumerate}
	\setlength{\parskip}{0\baselineskip} % 设置段间距
	\item Exploring the factors influencing walking duration to transit stations.
	\item Exploring the factors influencing transit ridership at station-level.
	\item Exploring the factors influencing transit ridership at station-to-station level.
	\setlength{\parskip}{0.7\baselineskip} % 设置段间距
\end{enumerate}

%
\section{Literature review} 
% 首先说明该研究通常包含三个点,然后贴那个大表,从三个点展开
Based on the research purpose and the collation of previous studies, rail transit ridership is commonly thought to be determined by the catchment area of stations and the environmental factors around the stations, and the accuracy of estimation on the factors' influence mainly depends on the choice of model. Therefore, the literature review is organized from the catchment area, method, influencing factor and respectively. Table \ref{tab:chp1:Review} gives a brief summary for the literature about transit ridership analysis.

% Table 1
% 这个表太大了,划线的时候使用toprule,bottomrule。否则格式会乱
\begin{sidewaystable}[htbp]
	\centering
	\caption{Summary of previous studies}
	\label{tab:chp1:Review}
	\scriptsize % 该表使用小字体
	\renewcommand{\arraystretch}{1.25} % 重设表间距
		
	\begin{spacing}{1} % 这中间行间距1倍
		% 这里的格式控制为限制单元格宽度,自动换行,并居中
		\begin{tabular}{|p{8em}<{\centering}|c|p{5em}<{\centering}|p{5em}<{\centering}|p{5em}<{\centering}|p{5em}<{\centering}|p{5em}<{\centering}|p{5em}<{\centering}|p{5em}<{\centering}|p{5em}<{\centering}|p{5em}<{\centering}|p{5em}<{\centering}|}
			\Xhline{1.5pt}
			\multicolumn{2}{|c|}{Year} & 2004 & 2004 & 2009  & 2010 & 2011 & 2012 & 2013 & 2013 & 2015  \\
			
			\Xhline{0.5pt}
			\multicolumn{2}{|c|}{Author} & Chu & Kuby et al. & Taylor et al. & Sohn and Shim & Gutiérrez et al. & Cardozo et al. & Chakraborty et al. & Zhao et al. & Jun et al.  \\
			
			\Xhline{0.5pt}
			\multicolumn{2}{|c|}{Catchment} & $1/4$ mile (400m) walking distance & Half mile (800m) walking distance & N/A & N/A & Distance-decay 800m buffer & 800m walking distance & N/A & 800m radius & 300m, 600m, 900m radius  \\
			
			\Xhline{0.5pt}
			\multicolumn{2}{|c|}{Method} & Poisson Regression & WLS & 2SLS & OLS, SEM & OLS & OLS, GWR & OLS, SEM & OLS & OLS, MGWR  \\
			
			\Xhline{0.5pt}
			\multicolumn{2}{|c|}{Sample Size} & 2568 & 268 & 265 & 251 & 158 & 190 & 900 & 55 & 442  \\
			
			\Xhline{0.5pt}
			\multicolumn{2}{|c|}{Number of Valid Indicator} & 15 & 11 & 8 & 7 & 9 & 4 & 9 & 11 & 11  \\
			
			\Xhline{0.5pt}
			\multicolumn{2}{|c|}{Coefficient of determination (Adjusted R2)} & 0.54 & 0.71 & 0.91 & 0.6 & 0.73 & 0.56 & 0.69 & 0.95 & 0.77  \\
			\Xhline{0.5pt}
			
			% 这里的multirow格式控制,限制宽度自动换行,内容部分加上\centering 可以居中
			\multirow{6}[0]{*}{\centering{Land-use factors}} & Building area & & & & $\bullet$ & $\bullet$ & & $\bullet$ & $\bullet$ & $\bullet$  \\
			\Xcline{2-11}{0.5pt}
			& Hospital & & & & & & & & $\bullet$ &  \\
			\Xcline{2-11}{0.5pt}
			& School/University & & & & $\bullet$ & & & & $\bullet$ &  \\
			\Xcline{2-11}{0.5pt}
			& CBD & & $\bullet$ & & & & & & $\bullet$ &  \\
			\Xcline{2-11}{0.5pt}
			& Land-use mix & & & & $\bullet$ & $\bullet$ & $\bullet$ & & & $\bullet$ \\
			\Xcline{2-11}{0.5pt}
			& Other infrastructures & & $\bullet$ & & & & & & $\bullet$ & $\bullet$ \\
			
			\Xhline{0.5pt}
			\multirow{6}[0]{8em}{\centering{Transit-related factors}} & Accessibility of pedestrian & $\bullet$ & & & & & & $\bullet$ & &  \\
			\Xcline{2-11}{0.5pt}
			& Accessibility of transfer & & $\bullet$ & & $\bullet$ & $\bullet$ & $\bullet$ & $\bullet$ & $\bullet$ & $\bullet$ \\
			\Xcline{2-11}{0.5pt}
			& Road coverage & & & $\bullet$ & $\bullet$ & & & & $\bullet$ &  \\
			\Xcline{2-11}{0.5pt}
			& Parking & & $\bullet$ & & & & & & $\bullet$ &  \\
			\Xcline{2-11}{0.5pt}
			& Service level of public transit & $\bullet$ & $\bullet$ & $\bullet$ & & $\bullet$ & $\bullet$ &       & $\bullet$ & $\bullet$ \\
			\Xcline{2-11}{0.5pt}
			& Locational factor & & $\bullet$ & & $\bullet$ & & & & $\bullet$ &  \\
			\Xhline{0.5pt}
			
			\multirow{8}[0]{8em}{\centering{Demographic and socioeconomic environment factors}} & Population & & $\bullet$ & $\bullet$ & $\bullet$ & & $\bullet$ & $\bullet$ & $\bullet$ & $\bullet$ \\
			\Xcline{2-11}{0.5pt}
			& Employment & $\bullet$ & $\bullet$ & $\bullet$ & $\bullet$ & $\bullet$ & $\bullet$ & $\bullet$ & $\bullet$ & $\bullet$ \\
			\Xcline{2-11}{0.5pt}
			& Age   & $\bullet$ & & $\bullet$ & & & & & & $\bullet$ \\
			\Xcline{2-11}{0.5pt}
			& Tenant proportion & & $\bullet$ & & & & & & & $\bullet$ \\
			\Xcline{2-11}{0.5pt}
			& Race  & $\bullet$ & $\bullet$ & $\bullet$ & & $\bullet$ & & & &  \\
			\Xcline{2-11}{0.5pt}
			& Income & $\bullet$ & & & & & & $\bullet$ & &  \\
			\Xcline{2-11}{0.5pt}
			& Vehicles holdings & $\bullet$ & & & & & $\bullet$ & $\bullet$ & &  \\
			\Xcline{2-11}{0.5pt}
			& Fare  & & & $\bullet$ & & & & & &  \\
			\Xhline{1.5pt}
		\end{tabular}
	\end{spacing}
\end{sidewaystable}

%
\subsection{Catchment area}
% historically,buffer area,PCA
An important precondition for investigating factors influencing transit ridership is the definition of the catchment area of a station, which has been interpreted in the part of research purpose. Historically, the determination of transit catchment area was mainly implemented in GIS by creating a distance buffer around the transit station \cite{o1992analysis,hsiao1997use,ayvalik2002heuristic,peng1997simultaneous}. However, since a transit catchment area is determined by the common walking distance from the transit station, which is also called pedestrian catchment area (Abbreviated as PCA), a circle buffer area is not adequate for presenting the catchment area. Recent decades, with the development in GIS and the richness in statistics, the application scenario of GIS has progressed far beyond the simple buffering operation \cite{biba2010new,wu2003ptal,jiang2012walk}. Also, many efforts have been made to estimate the real walking distance by importing the real road network. According to the existing studies, the PCA generally ranges from 400 meters to 1000 meters due to different station types, city forms, also travel preference \cite{alshalalfah2007case,guerra2012half,keijer2000people,murray1998public,o1996walking,zhao2003forecasting}.

% 这里给出支持800米, 然后这个800米其实也很不合理
Since difficulties in estimating the accuracy catchment area are still not well addressed, a general 800-meter (half-mile) walking distance, which has been verified to be suitable for most cases, has been widely accepted as a principal reference to determine the catchment area \cite{kuby2004factors,gutierrez2011transit,cardozo2012application,zhao2013influences}. This 800-meter, however, is just a general range of the catchment area for rail transit stations but not considering people's individual characteristics and travel preference. The catchment area is mainly determined on the base of questionnaires by asking how far/long did passengers walk to stations, or how far/long is passenger willing to walk to stations \cite{keijer2000people,zhao2003forecasting,garcia2013walking}, while the answers are usually given in a loose way like I remember I walked... or I think I prefer... What's more, the same reasoning has been used to determine the catchment areas of different stations even in different cities and countries. Till now, restricted by analytical method and data, there is no better option than this general 800-meter walking distance in determining rail transit catchment area.

%
\subsection{Method}
% 定位于交通预测模型
From the view of methodology, exploring and estimating the transit ridership can be treated as part of travel demand forecast \cite{miller1999potential,boyce1994introducing}. There are few fields in urban planning paying more attention to the statistical model for looking into the future than transportation. To date, extensive models have been developed and practiced for the issue of transit ridership, of which the most widely used are the activity-based four-step model and the direct model \cite{mcnally2007four,ewing2010travel}. While either of them works on travel demand forecast, they have different application scenario.

% 吐槽四阶段
Traffic Analysis Zones (TAZs), which is a definition put forward in four-step models, range from the scale of blocks to census tracts, commonly working at macro scales: corridors, subregions, and stats. The resolution of four-step models tends to be too gross to deal with the issues at neighborhood-scale \cite{cervero2006alternative}. Even though the four-step model has enjoyed widespread support from decades of use, it was never meant to predict the travel demand at a neighborhood-scale taking transit stations as cores, not to mention the estimation of influencing factors of the transit ridership \cite{cervero2006alternative,chu2004ridership,duduta2013direct}. Additionally, the trip generation stage in the four-step approach is usually conducted on the base of empirical model by using socioeconomic variables like population, employment, auto ownership. However, the impact of traffic indicators will not be exactly the same in different cases \cite{jones1983demand}. It is also unclear that whether these selected indicators significantly relate to traffic volume. linear relationship between traffic volume and these selected indicators. 

% 狂吹direct model
Direct models estimate ridership as a function of station environments and transit service features based on observed ridership and statistical data \cite{cervero2006alternative}. With the development of geographic information system (GIS) technology and the richness of digital statistic, direct model has gained popularity in estimating travel demand at neighborhood-scale. The advantages of direct models firstly stem from the ease of estimation. The only critical requirements are GIS, statistics for built-environments around stations, and the corresponding transit ridership \cite{guerra2012half}. The accuracy of prediction is also one of the advantages. It is easier and faster than other travel demand models, meanwhile, it can also provide strong predictive power \cite{lane2006sketch}. Compared with the disadvantage mentioned in four-step models, since the estimation of direct models is based on the historical statistics, it can justify the importance and validity of the indicators expected to have influence on the transit ridership \cite{walters2003forecasting}.

%
\subsection{Influencing factor}
% 影响transit ridership的因素分为了3类
Transit ridership is generally thought to be related to land-use, transportation environment, or travel preferences \cite{thompson1997achieving}. In 1997, Cervero proposed a three dimension index system (Density, Diversity, and Design) to examine the ridership of transit \cite{cervero1997travel}, which has been generally accepted as a basic principle. In addition, many extensions have also been added to the 3D theory, such as accessibility to the station, connectivity of line, and capacity of station \cite{beimborn2003accessibility,garcia2013walking}. In this study, all the candidate factors expected to influence transit ridership can be classified into three main categories: a. land-use factors; b. transit-related factors; c. demographic and socioeconomic environment factors.

% land-use factors
Land-use includes the buildings or facilities that provide the places for human activity, and it has been widely proved to have a strong relationship to ridership. Also, land-use diversity has a significant effect on ridership since it reflects the balance between traffic demands and supply in the catchment area. Although the definitions of land-use diversity are not the same according to different researchers, it is widely accepted that higher diversity tends to result in less transit ridership \cite{cardozo2012application,choi2012analysis,gutierrez2011transit,jun2015land,sohn2010factors,sung2014exploring}.

% transit-related factors
Transit-related factors are important to passengers going to take public transit. Better accessibility is thought to be attractive to passengers that are living far from the stations. The factors of accessibility are commonly described as the number of transfers, network density, number of parking facilities and walking convenience \cite{kuby2004factors,sohn2010factors,taylor2009nature,zhao2014analysis,chu2004ridership}. Also, the type and location of a station can affect accessibility as well. Terminal stations are thought to be more attractive. Because a terminal station is easier to transfer to other line or another mode of transportation \cite{o1996walking}.

% demographic and socioeconomic environment factors
The demographic and socioeconomic environment is an important factor which can reflect the travel preference. Obviously, the resident population and employment-population within the catchment area are crucial factors on ridership of the subway station. Besides, the economic factors also play an important role in ridership. For example, if the car ownership is higher in the catchment area of a rail transit station, people living there are more likely to choose private car than public transit \cite{chiou2015factors,zhao2005transit}. A higher percentage of low-income household can lead to an increase in the utilization of public transit \cite{thompson2012really}. Furthermore, the ratio of apartments and rental house within catchment have been verified being relevant with ridership in some degree \cite{jun2015land}.

%
\section{Primary research questions}
% 该领域研究大方向的问题
% 针对三个方面的综述,说明每个方面的问题
A lot of progress has been made in the research field of transit ridership forecasting, however, some work is still inadequate in the specific fields of the catchment area, model and influencing factors. Till now, even though there have been a lot of research and investigation on catchment areas, the key problems are still not well addressed. The only thing that can be determined is a general scope of catchment area which is from 400 meters to 1000 meters, but this scope varies too much in terms of different cases. A comprehensive and systematic methodology for estimating catchment area is needed, either the research perspective or the methodology needs to be further discussed. For the topic of forecasting rail transit ridership, most studies are keeping on mining the problems at station-level, but rarely exploring the problems from perspectives of rail transit lines and networks. For the estimation model, little progress has been made in the direct model so far. Most existing studies are conducted based on the regression model but without essential progress. As to influencing factors, scholars have done a thorough job of summary and classification. Nevertheless, factors behave differently in different cases, the approaches to select valid factors of specific cases are still insufficient. 

% 本文将针对的研究点
On the base of fully understanding the existing literature, this dissertation will be expanded upon 4 specific research points corresponding to the main purpose.

% 描述的再详细一点
\begin{enumerate}
	\setlength{\parskip}{0\baselineskip} % 设置段间距
	\item Exploring the influencing factors of the walking duration to transit stations in terms of passengers' personal attributes. 
	\item Summarizing and analyzing the characteristics of rail transit ridership, land-use, population etc. using the case of Fukuoka city.
	\item Exploring the factors influencing transit ridership at station-level with the small sample study case of Fukuoka. 
	\item Exploring the factors influencing transit ridership at station-to-station level.
	\setlength{\parskip}{0.7\baselineskip} % 设置段间距
\end{enumerate}

%
\section{Dissertation organization}
% introduction for study case
The dissertation uses the rail transit system of Fukuoka City, which is the fifth largest city in Japan having a more than 1.5 million population, as the study case. The rail transit system of Fukuoka has 78 rail transit stations, including 27 JR Kyushu stations, 35 subway stations, and 16 West Japan Railway stations. Figure \ref{fig:chp1:ResearchArea} presents the research area and the distribution of rail transit stations. The rail transit system of Fukuoka carries a daily average of more than 0.6 million passengers by 2015 that accounting for about 20\% in total motorized travel (Ministry of Land, Infrastructure n.d.). The rail transit system of Fukuoka is not a large-scale one, the share ratio of rail transit is staying at a lower level. Like many local central cities in Japan, Fukuoka city also has the problems in urban traffic congestion and financial press on public transit due to the tendency of using private cars and the huge operating costs on the subway system. Improving subway ridership has been an urgent demand for the Fukuoka government.

% Figure 2
\begin{figure}[htbp]
	\centering
	\includegraphics[width=\linewidth]{chapter1/ResearchArea}
	\caption{Research Area}
	\label{fig:chp1:ResearchArea}
\end{figure}

% organization of dissertation
The content of this dissertation is arranged as follow. Figure \ref{fig:chp1:Flowchart} gives a brief flowchart of the organization.

\begin{itemize}
	\item \emph{\textbf{Chapter 1}} gives a general introduction to the entire logic and flow with the support of a comprehensive literature review.
	\item \emph{\textbf{Chapter 2}} explores the correlation between passengers' personal attributes and surveyed walking duration to rail transit station.
	\item \emph{\textbf{Chapter 3}} clarifies the characteristics of transit ridership and land-use, and explores the overall relationship between them.
	\item \emph{\textbf{Chapter 4}} explores the factors influencing transit ridership at station-level focusing on some key questions in the small sample case.
	\item \emph{\textbf{Chapter 5}} explains the influence of land-use patterns on transit ridership at station-to-station level.
	\item \emph{\textbf{Chapter 6}} makes a conclusion of the three research point, and prospects the next stage of this research.
\end{itemize}

% Figure 1
\begin{figure}[htbp]
	\centering
	\includegraphics[width=0.8\linewidth]{chapter1/Flowchart}
	\caption{Flowchart of the Dissertation}
	\label{fig:chp1:Flowchart}
\end{figure}

% 章节划分
\clearpage % 新起一页
\bibliographystyle{apacite}
% \bibliographystyle{IEEEtran}
% 这里需要指定main文件所在路径的相对路径,否则无法读取参考文献。多个bib文件用comma分隔,没有空格
\bibliography{ref}