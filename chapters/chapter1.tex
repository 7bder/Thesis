\chapter{General Introduction}
\markright{CHAPTER 1}

%
\section{Issues in reality}
% status quo
With the continued urbanization, people continue to migrate from rural areas to urban areas. As a result of this, 54 percent of people worldwide live in urban areas by 2014, this population shift will keep going and is predicted to reach 66\% by 2050 \cite{UN2014world}. Living in cities is considered much more efficient and convenient than that in rural areas in a variety of ways, as it is easier to provide services when people live closer together. However, cities also change the way that humans interact with each other and the environment, often causing multiple problems. The second UN Conference on Human Settlements in 1996 came to the conclusion that the cities all over the world are facing problems due to urbanization.

% Different situations in developing countries and developed conutries
Although the problems commonly exist all over the world, the type and scale of problems differ from different stages of urbanization. As to developing countries who are now experiencing a rapid growth in the urbanization, the problems mainly include such as traffic congestion, disorganization, and air pollution, which usually follow by the rapid migrants from rural areas to urban areas. Essentially, most of the problems are caused by the unbalance of demand and supply in land-use, resource, and infrastructure. How to satisfy this demand is an important issue that the governments in developing countries have to consider and address. Different from the situation in developing countries, population aging and low birthrate has become the serious problems in developed countries. With a rapidly increasing proportion of old people in the population, governments are forced to increase expenditure on social security, adding to that the low birthrate, further intensifies the shrinking of working-age population, thus increasing the fiscal burden of governments. How to reduce the public financial expenditure and improve the efficiency of social operation has become a problem that the governments in developed countries have to face. 

% Summary of status quo
In some ways, the problems of the developing countries and the developed countries are very different, but in essence, either of them is the problem of sustainability. While governments in the developing world must try to increase the supply of services to satisfy the increasing population, those in the developed world must try to cope with economic slowdown due to decentralization and changing working patterns. Under such a realistic background, a sustainable program for urban planning seems to be necessary.

% Introduce TOD
In urban planning, transit-oriented development (TOD) is a type of urban development that mix residential, business and leisure space within walking distance of public transport, which was first to be proposed by Calthorpe, Peter in 1993 \cite{calthorpe1993next}. In so doing, TOD aims to increase public transport ridership and by reducing automobile travel, thus promoting sustainable urban growth. As to the precise definition of TOD, while the details vary in scope and specificity, most TOD definitions share several common elements \cite{boarnet1997story,bernick1997transit,megally2001california,cervero2004transit}:

\begin{itemize}
	\item Mixed-use development
	\item Transit stations as cores
	\item Compactness
	\item Pedestrian-friendly
\end{itemize}

% 
Now the guidelines of TOD summarized above has gained popularity by all levels of governments as the means of mitigating most of the common urban problems, including traffic congestion, air pollution, and incessant sprawl \cite{cervero2002transit}. In a common sense, an area based on TOD typically has a central transit stop (such as a train station, or light rail or bus stop) surrounded by a high-density mixed-use area, with lower-density areas spreading out from this center. The densest areas of a TOD are normally located within a radius of 400 meters to 800 meters (5 minutes to 10 minutes walking duration, varying by different kinds of public transit) around the central transit stop, as this is considered to be an appropriate scale for pedestrians, thus solving the last mile problem. 

% Change the objective to rail transit station.
% Interpret the importance of ridership prediction
% 这两个可以合并
% 随着城市规模的扩大以及技术的发展,轨道交通在现代城市中得到了极大的普及。越来越多的TOD也是基于轨道交通站形成的。
Urban rail transit is a type of high-capacity public transport, which has become an important travel mode in modern cities because of the rapid, punctual and environment-friendly features. 




% 

\section{Purposes of the dissertation}
% 我要预测客流
% 阐述一下客流预测的主要方法,最终在TOD下,我选择了station-level的预测
% station-level的客流预测可以分为几大步骤,存在的问题点

\section{Issues in research}
% 分问题点的review


\bibliographystyle{plain}
% 这里需要指定main文件所在路径的相对路径,否则无法读取参考文献
\bibliography{chapters/ref_INTRO}