\chapter{Introduction}
\section{AIJ}
\subsection{Introduction}
With the problem of weakness in population growth and the tendency of using private transport in the local central city of Japan, many operators of public transport in Japan, especially urban rail transit such as subway are now facing financial pressure due to the huge operating costs. How to increase the use of public transportation has become an important issue for the government of the local central city in Japan, and in order to turn around the bad financial situation(1), many efforts have been made to attract public transit user. To increase the use of public transportation, understanding how various factors affect the ridership is treated as a foundation for the policymaker.

From the view of methodology, exploring and estimating the factors influencing subway ridership is treated as part of traffic demand forecast(2)(3). In the past, limited by data and analytical tools, the traditional four-step approach was usually applied to the issue of macro traffic forecast, rather than exploring and estimating the factors influencing transit ridership(6,5,4). In the four-step approach, trip generation stage is usually conducted based on an empirical model using some socioeconomic variables like population, employment, auto ownership. However, in different city cases, the impact of traffic indicators will not be exactly the same(7). Additionally, it is also unclear whether these selected indicators are significantly related to traffic volume, or whether there is an indeed linear relationship between traffic volume and these selected indicators.
With the development of GIS technology and the richness of digital statistic, now extracting more accurate data in the catchment area of a station has become possible, thereby exploring and estimating the factors having significant relationships with ridership(8), which formed the so-called direct station-level ridership forecasting model.

This study can be viewed as an extension of existing station-level ridership model for a small sample case. Different from the case with hundreds of samples, a small sample case with dozens of stations has a higher risk of both type I and type II errors in statistic when identifying the valid variables that should be entered the regression model. For this problem, the aim of this study can be stated as: exploring and explaining the factors influencing subway ridership using the small sample case. The main work includes 3 aspects: a. to build the index framework based on prior study; b. to identify the valid indexes that do affect subway ridership; c. to explain the relationship among variables in generating subway ridership quantitatively.

\subsection{Literature Review}
2. Literature review
There are 3 main research points in this kind of issue: a. How to determine the catchment area; b. How to construct the index framework; c. How to choose and estimate the mathematical model.

2.1 Catchment area
An important assumption for investigating factors influencing transit ridership is the definition of the catchment area of a station. The catchment area is defined as the influence area of a station, which represents the maximum distance that most people can accept to walk to the station. Because it is determined by the real walking distance of pedestrian along the street network, the catchment area is also called pedestrian catchment area (Abbreviated as PCA). In the previous studies, the thresholds were generally considered ranging from 400m to 1000m(14,13,12,11,10,9), in which the 800m distance is the most accepted one. However, the 800m threshold cannot be considered as a standard one, because of the different features of land use and demography in different cities.

Many studies also found that the greater the distance to stations within the catchment area is, the less public transit tends to be used. Therefore, some studies began to use the function of distance-decay to calculate the variables within catchment area(15). The distance-decay method now is gradually becoming accepted and it has been proved to be closer to fact. Although this distance-decay method seems to be able to provide more accuracy result, it also has some limitations, and it cannot work all the same for different study cases. One of the main reasons is, to achieve the correct distance-decay function, a large-scale personal trip survey is necessary. In addition, since the probability for people to choose the transit system is hypothesized having a linear relationship with the walking distance, the method of distance-decay and distance threshold is regarded as equivalent when the distribution of the population in the catchment area is not significant in spatial auto-correlation.

2.2 Factors
In the field of urban planning, transit ridership is generally thought to be related to land use, transportation environment, or travel preferences(16). In 1997, Kockelman proposed a three dimension index system (Density, Diversity, and Design) to examine the ridership of transit(17), which has been generally accepted as a basic principle. In addition, many extensions have also been added to the 3D theory, such as accessibility to the station, connectivity of line, and capacity of station(19,18). In this study, all the candidate factors expected to influence transit ridership can be classified into three main categories: a. land use factors; b. transit-related factors; c. demographic and socioeconomic environment factors.

Land use includes the buildings or facilities that provide the setting for human activity, and it has been widely proved to have a strong relationship with ridership. Also, land use diversity has a significant effect on ridership since it reflects the balance between traffic demand and supply within the catchment area. Although the definitions of land use diversity are not the same according to different researchers, it is widely accepted that higher diversity tends to result in less transit ridership(24,23,22,21,20,15).

Transit-related factors are important for passengers going to take public transit. Better accessibility is thought to be attractive for passengers living further. The factors for accessibility are commonly described as the number of transfers(25,20), network density(27,26), number of parking facilities(27,25) and walking convenience(4), etc. Also, the type and location of a station can affect accessibility as well. Terminal stations are more attractive for passengers because people can accept to spend more time on getting to a terminal station which is easier to transfer to other line or another mode of transportation(13).

The demographic and socioeconomic environment is an important factor which can reflect the travel preference. Obviously, the resident population and employment-population within the catchment area are crucial factors on ridership of the subway station. Besides, the economic factors also play an important role in ridership. For example, in the area where the car ownership is higher, people are more likely to choose private car than public transit(29,28); also the higher the percentage of low-income household is, the more likely people tend to take public transit(30). Furthermore, the ratio of apartments and rental house within catchment have been verified being relevant with ridership in some degree(23).

2.3 Model
To estimate the relevance of various factors and ridership, the model of linear regression is a widely adopted one(31,17). However, due to the insufficient consideration of heteroscedasticity and spatial autocorrelation, the result of regression often leads to large standard errors or low level of significance. Thus, linear regression model is not available to all the stations with different characteristics. To improve the generality of the model, the extension such as Geographically Weighted Regression (GWR), Weighted Least Squares Regression (WLS) and Poisson Regression etc. have been successively introduced into the issue of direct ridership model.

Chu and Choi et al. estimated the ridership of bus using the model of Poisson Regression(22,4). The problems occurred in ordinary linear regression such as the contravention between fact and estimated coefficients, and the low level of significance was well addressed. Therefore, both generality and explanatory ability in the regression model were enhanced.

Some studies have achieved a high coefficient of determination at the first stage with OLS, however, the result showed a significant heteroscedasticity and a non-random distribution of estimated residual(25). To deal with this deviation at the second stage, WLS was brought in to eliminate heteroscedasticity, in which the data points were weighted using the standard error. The result showed that WLS was effective in eliminating heteroscedasticity and improving the explanatory ability of the model.

The data points in OLS are regarded to be independent of each other, however, each data point has different geographical location in the issue of direct ridership model, the observed values are not considered to be independent of each other in terms of the fact that they are distributed continuously in space. For one data point in regression, the observed value is related to the data point nearby in geographical location, and the regression parameters in different geographical locations usually have different performances in their characteristics(32). For the problems of spatial autocorrelation and spatial heterogeneity, Cardozo made a comparison of OLS and GWR with the same regression parameters, the result showed that the coefficient of determination had a significant improvement and the standard errors turned to be less in GWR(21). On the basis of common GWR, Jun and Zhao introduced Mixed Geographically Weighted Regression (MGWR) to this issue in the consideration of that some regression parameters did not have special autocorrelation(28,23). They set part of the parameters as global independent variables, and the others as spatially autocorrelated variables, to make model closer to fact.

2.4 Summary
Although existing studies have done a lot on the issue of direct ridership model, there is still some insufficiency in each study due to the limitation of study case and data source, especially for small sample case. For the selection and construction of factors, the simple and direct indicators such as population, employment etc. are roughly the same with those in the existing studies. But the definition of indicators obtained by secondary calculating such as land use diversity, bus service etc. are not the same, and the effects of such factors have been neither well verified nor widely accepted. For the model, most of the coefficient of determination in OLS were not very ideal (less than 0.7), there was still more than 30\% of the change in ridership not being explained by the model(23,15). Additionally, even though some of the studies have obtained the high coefficient of determination exactly, there was still another problem that one factor had too strong effect while the rest of the factors had very little influence on the ridership.

This study will focus on the small sample case and try to address some of the shortcomings stated above. The main work of this study can be divided into 3 parts. 1) Build the index framework based on the previous studies, and proposes new indicators to help describe the variation in the subway ridership of Fukuoka. 2) Optimize the procedure of identifying a valid explanatory variable, thereby making it applicable for small sample cases. 3) Improve the accuracy in the estimation of the regression model in the terms of small sample cases.

\section{CUPUM}
Urban rail transit is a type of high-capacity public transport general-ly found in urban areas. Because of the rapid, punctual and environ-ment-friendly features, the urban rail transit is becoming one of the most important travel modes in modern cities. With the popularity of the urban rail transit in modern cities and the emphasis on sustainable development, the concept of TOD (Transit-Oriented Development) is put forward, intended to build the compact city (Calthorpe 1993). Based on the perspective of TOD, many cities around the world have adopted the policy of giving priority to the development of public transport for decreasing the share of motorized travel and increasing the willingness of using public transit. For policymakers, how to grasp the relationship between land-use and passenger volume has become an important issue that they must face.
Until now, there are many studies focusing on the relationship be-tween various factors and transit ridership from the perspective of TOD. Most of the studies are based on regression-type model and conducted from the view of station-level, the transit ridership is thought to be affected by the circumstance surrounding the station (Cervero \& Kockelman 1997; Taylor et al. 2003; Zhao et al. 2005; Estupinan \& Rodriguez 2008; Taylor et al. 2009; Sohn \& Shim 2010; Gutiérrez et al. 2011; Jun et al. 2015). Among them, the multiple line-ar regression models are the earliest and most widely used model (Cervero \& Kockelman 1997; Gutiérrez et al. 2011). However, the da-ta point in ordinary least squares (OLS) model is treated as a single point, which is not consistent with fact since the transit node is con-nected to each other. To deal with the relationship among stations in the network, the approach of spatial regression is also introduced into this issue (Cardozo et al. 2012; Jun et al. 2015). However, this rela-tionship among stations in spatial regression models is just the expres-sion for the distribution relationship of stations in location, it cannot reflect the real connectivity between two station areas.

To explore the connectivity between station areas, Choi et al. con-ducted a station-to-station-level investigation into the effect of both origins and destinations on OD metro ridership of Seoul, Korea by us-ing the data from the automatic fare collection system (Choi et al. 2012). The factors considered to influence the OD ridership is divided into three groups: the factors of both origin and destination, and the impedance factors between stations. And where the variables of origin and destination are the same, representing the travel characteristics of O and D respectively. The influence of factors on OD ridership is es-timated using multiplicative and Poisson regression, with the data of morning, evening peak hours, and midday hours. This station-to-station approach has connected stations by using the factors of both origin and destination. As the result of this empirical study, different land-use functions have different travel characteristics in terms of both time and space. However, this approach still cannot reflect the connectivity between two station areas because of the aggregate pro-cessing for data.

Land-use and public transit are coevolving partners in city building (Handy 2005; Dittmar \& Ohland 2012). In the urban railway transit system, the ridership between stations is thought to be related to land-use, distribution of functional regions, or travel preferences (Thomp-son 1997). In urban planning, TOD can be viewed as a method for balancing the land-use of residences, business, and leisure within walking distance taking the station as the center, while the transit rid-ership between stations can be viewed as the connectivity of different TOD areas. The TOD area is generally referring to a compact residen-tial district that includes mixing land-use to allow people to most of their daily activities within the easy walking distance of a major transit node (Lund et al. 2004). In details, various functional buildings are the carrier for people to live, work and recreate, different functions of buildings correspond different trip purposes. When the functions of buildings within the easy walking distance of a station cannot satisfy the requirement of people’s daily activities, people will choose to go to other places to conduct their business by using the transit node such as the subway. Therefore, the distribution of different functional building in a TOD area is considered to not only affect the ridership of the station where it is located but also affect the ridership of other sta-tions connected to that station.

With the goal of explaining the variation in the ridership between stations, this study will focus on the passengers’ choices for destina-tion stations from the perspective of complementarity of building function in different TOD area, using the case of subway network in Fukuoka, Japan. The result from this study also can provide a founda-tion for explaining the connectivity between different TOD areas. Fac-tors that are expected to influence the connection of stations are stated in the next section. Then the approach and model that are adopted in this study are interpreted. Based on the approach and model, the case of the subway system in Fukuoka, Japan is investigated. The discus-sion for the result is also presented in the last section.

\section{AURG}
\subsection{Introduction}
Mass Rapid Transit (MRT) has been one of the most important parts in public transit system in modern cities. How to shift more people from private car user to public transit passenger is always a key topic for both urban planning and management. There are many factors that can influence the use of public transit, if considering from overall, they can be separated into two parts which are the convenience of public transit (environment factors) and the willingness of traveling by public transit (behavior factors). This study will focus on the behavior factors to examine the relationship between passengers' individual characteristics and walking durations from departures to transit stations.

%
At present, the 800m (half-mile) walking distance has been widely accepted as a principal reference of the catchment area for the planning of Transit-Oriented Development (TOD). Planners and researchers also use transit catchment areas to make prediction of the transit ridership. This 800m walking distance is loosely obtained from the sampling survey by asking how far people are willing to walk to transit stations, but the same reasoning has been used to justify other transit catchment areas and even in different cities and countries. People with different individual characteristics generally should have different preferences of walking duration, even if they have the same preference of walking duration, their walking distance is not the same because of different walking speeds. It can be assumed that potential public transit users have their own preference and tendency for a given threshold of walking duration. The willingness of using public transit will raise if the expected walking duration is less than the acceptable walking duration, otherwise, the willingness will decrease. Even though the surveyed walking durations should not be viewed as the direct reflection of passengers' willingness for walking durations, they can reflect the differences in the willingness of passengers who have different individual characteristics in an indirect way. On the basis of above-mentioned, this study attempts to give explanations on how individual characteristics influence the walking duration using the study case of Fukuoka subway. The expected result of this study can be viewed as a reflection of willingness for walking durations, and also is expected to provide references for helping determine the catchment area of transit stations.
\subsection{Literature Review}
This section reviews the literature on the issue of walking duration/distance to transit stations, some problems that still not well addressed are collated and summarized. The review is arranged into two parts, how to deal with the variable of walking duration (research object in this study), and how to estimate the feature of individual characteristics (independent variables in this study).

%
For the walking duration, the research object in this study, there is always a difficult point in obtaining the accuracy walking distance/duration by questionnaire because of the discrepancy between perceived values and objective values \cite{BadlandHannahMandSchofieldGrantMandSchluter2007,McCormack2008}. Also, the observed walking distance/duration is not the reflection of how long people are willing to spend on walking to transit stations, but only the walking distance/duration between the departure and the transit station. For this problem, some studies chose a different perspective trying to explain the walking distance/duration by introducing one threshold of walking duration \cite{Besser2005,McCormack2008}. They examined the differences in the distribution of influencing factors over/under the specific threshold of walking duration. Indeed, using a threshold can decrease the discrepancy between observation and reality in some extent, but this disposal also brought some new problems in, for example, it may lead to a great loss of information in the raw data, and it is also difficult to decide the threshold of walking duration. Moreover, another point for this issue is whether to choose walking distance or walking duration as the threshold. To date, there have been a lot of studies working on the relationship between walking distance and passengers' individual characteristics, some of them argued that an individual has a limited amount of time spending on traveling during a day, people tend to accept further walking distance as the speed of travel increases \cite{Marchetti1994,Larsen2010}. That means, passengers with the same individual characteristics may have the similar willingness of walking duration, but they generally have different willingness of walking distance due to different travel speed. This is also the reason why this study chooses the walking duration as the research object.

%
In the influencing factors for walking durations, passengers' individual characteristics are generally thought to be the key that can affect walking distance \cite{Besser2005,WeinsteinAgrawal2008,Krygsman2004,Yang2012,Daniels2013,Guerra2012}, whereas, there are few studies having clearly verified the relationship between individual characteristics and walking distance/duration, even there is a study suggesting the walking distance should not be viewed as a function of socio-demographic characteristics \cite{Krygsman2004}. Several studies have confirmed the role of travel purposes in determining walking distance, the commute trip showed particularity from the other purposes, people with the purpose of commute tend to walk a longer distance to go to transit stations \cite{Larsen2010}. However, the definite relationship between trip purposes and walking distance is still unclear. The same situation is for other categories of factors, such as the factors of transportation environment, land use, and willingness of passengers \cite{Guerra2012,Krygsman2004,WeinsteinAgrawal2008}. The only thing that has been confirmed to date is that the walking distance/duration can be influenced by some specific kinds of factors, such as socio-demographic characteristics, trip purposes, and built-environment, but the problem is how and how much the walking distance/duration can be influenced.

%
In summary, perhaps because of the problems in either the research object or influencing factors, most of the existing studies did not find the significant relevance between the walking distance/duration (research object) and the influencing factors (independent variables). The studies working on the qualitative description for the distribution of walking distance accounted for the majority, although some of the existing studies attempted regression model on this issue, the desired results were not obtained. To avoid such problems mentioned above, this study uses the thresholds of walking duration as the research object. For any given threshold of walking duration, the respondent who gives the answer greater than the given threshold can be viewed as this respondent can accept this threshold of walking duration. The relationship between walking durations and individual characteristics is explained by examining the feature distribution over/above the given thresholds of walking duration. For the analytical method, since various types of regression model has been verified not suitable for this issue, instead of finding the linear relevance between dependent and independent variables, this study introduces the approach of machine learning into this issue, and try to explain the relationship between walking durations and individual characteristics from the view of probability.

%\bibliographystyle{plain}
% 这里需要指定main文件所在路径的相对路径,否则无法读取参考文献
%\bibliography{reference}