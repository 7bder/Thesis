\chapter{Analysis on the characteristics of transit ridership and land-use}
\markright{CHAPTER 3}
%
\section{Introduction}
%
\subsection{Background}
%
In recent years, the problem of aging population has occurred in many developed countries, which also always accompanied with decline in population. As a local central city, Fukuoka now is still in the population growth period, the population has reached 1.5 million, nevertheless, the proportion of aging population is continuously increasing as well. According to the census data, it is expected that the population will reach the peak in 15 years and shift to the population decline period, moreover, the aging population will be more than one quarter of the total after 10 years (as shown in Figure \ref{fig:chp3:PopulationChange}). On the other hand, the data of Kitakyushu Person Trip Survey shows an inclination of that the private car share rate will keep on increasing while the rail transit share rate will turn to decrease in future (refer to Figure \ref{fig:chp3:TrafficModeShare}), As a result, this trend of the shift in population structure and traffic share rate will lead to a decrease in the income and an increase in financial pressure in the case of Fukuoka. The same problem will also occur in most of the local central cities similar with Fukuoka. Addition to the financial problem, traffic congestion is also becoming the problem to all the resident. Figure \ref{fig:chp3:TravelSpeed}, which is quoted from Road Traffic Census (2010), shows the average travel speed during crowd time for the major cities of Japan. As we can know from this figure, the problem of traffic congestion is becoming more and more serious for the downtown area of Fukuoka.

%
\begin{figure}[htbp]
	\centering
	\includegraphics[width=\linewidth]{chapter3/PopulationChange}
	\caption{Trends of population change}
	\label{fig:chp3:PopulationChange}
\end{figure}

%
\begin{figure}[htbp]
	\centering
	\includegraphics[width=\linewidth]{chapter3/TrafficModeShare}
	\caption{Traffic mode share rate}
	\label{fig:chp3:TrafficModeShare}
\end{figure}

%
\begin{figure}[htbp]
	\centering
	\includegraphics[width=\linewidth]{chapter3/TravelSpeed}
	\caption{Private car travel speed ranking}
	\label{fig:chp3:TravelSpeed}
\end{figure}

%
According to the situation stated above, obviously, the issue put in front of the local central cities like Fukuoka is how to promote the use of rail transit, thus to reverse the financial dilemma of rail transit operator and making a better living environment for the resident. To achieve this goal, it is necessary to make clear what factors can influence the rail transit ridership, based on which to make new policy helping improve the role of rail transit.

%
\subsection{Previous studies}
%
Many works have been done on the topic of rail transit ridership and the environment around transit stations, while most of them focused on the trend of variation in rail transit ridership or land-use, concentration on the studies of the relationship between the transit ridership and land-use is still inadequate \cite{matsumoto2013study,nakamura2015study}. 

%
Depending on the research purpose, the research scale is also different. For example, the study on the changing trends in transit ridership at the scale of Shinkansen mainly aims at making clear the role of each city from the view of entire country \cite{matsumoto2013study}. At a relative small research scale, for example, some studies focused on the urban rail transit within metropolitan area to make clear about the changing trends in urban structure \cite{song2013evaluation,baba2012change}. This study further narrows the research scale to the urban rail transit within cities. It mainly focuses on making clear about the changing trends in rail transit ridership itself, thus providing reference to making policies of urban planning and management \cite{nakamura2015study,yano2008}. 

%
land-use around transit stations is always thought as the key factor influencing the transit ridership, while on the other hand, land-use is also thought to be affected by the transit station. The changes in distribution and types of shop around transit stations are generally considered as good ways to reflect the influence on land-use impacted by the stations, the conclusion drawn from existing studies also supported this argument for that the changes of shops around transit stations showed clear characteristics \cite{sui2013research,zhao2012study,kitayama2008study}. The other kinds of facilities belong to different land-use also relate to rail transit ridership, such as clinic, school, and some other public facilities \cite{lee1995predicting,lee1994temporal}. 

%
Some studies also analyzed the influence on transit ridership from the perspective of land-use. A study on the changing trend of transit ridership gave the main conclusion that mixed land-use around rail transit stations has a constant effect on increasing transit ridership \cite{nakamura2015study}. Quantitative analysis on the relationship between transit ridership and land-use usually conducted using regression model (refer to Table \ref{tab:chp1:Review}), which is also applied to the case of Japanese cities. A study using the case of rail transit stations within Tokyo metropolitan indicated that the land-use factors of residence, office, and education have significant influences on rail transit ridership \cite{tadakatsu2015empirical}.

%
Land-use is widely accepted as one of the most important factors influencing transit ridership, nevertheless, the problem is how to find the specific factors of land-use to estimate transit ridership, and how to evaluate this effect quantitatively and precisely. Besides, the influencing factors are not only land-use, some other factors such as road network, floor area ratio, transfer structure etc. also have an interactive relationship to rail transit ridership \cite{kondo2010railway,inohae2009study}. There are also many factors, even though which has not been fully confirmed yet, maybe also play the important role in determining transit ridership.

%
\subsection{Research purpose}
%
With the goal of promoting the use of rail transit, this chapter focuses on making clear about the characteristics of annual change in rail transit ridership and the land-use around the transit stations. Then Based on the characteristics to explore the relationship between transit ridership and land-use. 

%
Specifically, the research has two main purposes: 
\begin{enumerate}
	\setlength{\parskip}{0\baselineskip} % 设置段间距
	\item To describe the characteristics of transit stations in terms of both transit ridership and land-use.  
	\item To explore the relationship between transit ridership and land-use on the base of intensive description on the characteristics of transit stations. 
	\setlength{\parskip}{0.7\baselineskip} % 设置段间距
\end{enumerate}

%
\subsection{Research objects}
%
The research objects of this study are the 35 subway stations of Fukuoka, and several reasons are given here for doing this:

%
\begin{enumerate}
	\setlength{\parskip}{0\baselineskip} % 设置段间距
	\item The catchment area of subway stations is covering all the downtown area of Fukuoka, and most of the urban area. 
	\item Subway system undertakes the major rail transit traffic within the urban area of Fukuoka. 
	\item Since subway system is not serving the traffic of intercity, the influencing factors of transit ridership can be easily confined to the area around transit stations. It is conductive to the analysis on the relationship between transit ridership and land-use.
	\setlength{\parskip}{0.7\baselineskip} % 设置段间距
\end{enumerate}

%
This study works on the transit ridership and the land-use around transit stations, but how to define the area of “around”? Since this study is not aiming at predicting transit ridership using land-use factors, the main purpose is to understand the characteristics of transit ridership and land-use around the stations. On the base of understanding the characteristics, then to explore the relationship between land-use factors and rail transit ridership.



%
\section{Data}
\subsection{Study case introduction}
%
The study case in this study is the subway system of Fukuoka, some details are shown in Table \ref{tab:chp3:SubwayLineInfo}. It consists of three subway lines, the Airport Line (Line 1), the Hakozaki Line (Line 2) and the Nanakuma Line (Line 3). The three lines are operated by the Fukuoka City Transportation Bureau, this subway system is not a large-scale one, which only has 35 stations in total. The distribution and names of the 35 stations are shown in Figure \ref{fig:chp3:SubwayStations}.

% Table
\begin{table}[htbp]
	\centering
	\caption{Information of Fukuoka subway stations}
	\label{tab:chp3:SubwayLineInfo}
	\small
	\renewcommand{\arraystretch}{1.25} % 重设表间距
	\begin{tabular}{ccp{5em}<{\centering}p{4em}<{\centering}p{4em}<{\raggedleft}p{3em}<{\centering}p{4em}<{\centering}}
		\Xhline{1.5pt}
		Line & \multicolumn{1}{c}{Name} & First section opened & Last extended & \multicolumn{1}{c}{Length} & Stations & Gauge \\
		
		\midrule
		1 & Kukou Line & 1981 & 1993 & 13.1 $km$ & 13 & 1067 $mm$\\
		2 & Hakozaki Line & 1982 & 1986 & 4.7 $km$ & 7 & 1067 $mm$ \\
		3 & Nanakuma Line & 2005 & - & 12.0 $km$ & 16 & 1435 $mm$ \\
		\multicolumn{2}{r}{Total} & - & - & 29.8 $km$ & 35 & - \\
		\Xhline{1.5pt}
	\end{tabular}
\end{table}

\begin{figure}[htbp]
	\centering
	\includegraphics[width=\linewidth]{chapter3/SubwayStations}
	\caption{Distribution of the subway stations}
	\label{fig:chp3:SubwayStations}
\end{figure}

%
From the spatial distribution, the subway stations covered most of the core area of Fukuoka, Line 1 and Line 2 are connected while Line 3 is separated from the other two lines. The catchment area of stations with the number of 8, 9, 10, 11, 20 covers the downtown area, where has a higher density of both population and building. Station 7 locates in Fukuoka airport, which mainly serves the airport. Station 11, Hakata Station, is a comprehensive railway transportation hub of Fukuoka integrating Shinkansen, JR, subway, and bus terminal. Station 10, Tenjin Station, is another central transportation hub, including West Japan Railway, subway, and bus terminal. The two transportation hubs undertake the role as the passenger distribution center not only within Fukuoka urban area but also extending to other cities around Fukuoka city. The endpoint stations of 11 and 16 connect to other railway lines extending to other cities.

%
\subsection{Data collection}
%
All the dataset used in this study comes from the official statistics, details for the dataset is listed in Table \ref{tab:chp3:DataSource}. The data onto subway transit and population is annual statistics, while the data onto urban planning basic survey is provided every five years. Since the subway line 3 is opened by 2005, the research period is set to the 10 years after the subway line 3 is operated. The statistics of population and land-use is on the accuracy of town-chome, which is the smallest region in Japanese administrative division. 

% Table
\begin{table}[htbp]
	\centering
	\caption{Data source}
	\label{tab:chp3:DataSource}%
	\small
	\renewcommand{\arraystretch}{1.25} % 重设表间距
	\begin{tabular}{llll}
		\Xhline{1.5pt}
		Item & Source & Data accuracy & Time point \\
		
		\midrule
		Subway transit ridership & Fukuoka Traffic Bureau & Station & 2005-2014 \\
		Population & Resident Basic Account & Town-chome & 2005-2014 \\
		Land-use & Urban Planning Basic Survey & Building & 2003, 2008, 2012 \\
		\Xhline{1.5pt}
	\end{tabular}%

\end{table}%

%
\subsection{Data preprocessing}
%
Since the data accuracy is not matched with population and land-use, the preprocess are necessary before conducting the analysis. The data preprocess can be separated to 3 aspects.

%
\begin{enumerate}
	\setlength{\parskip}{0\baselineskip} % 设置段间距
	\item Matching the range of data unit. The minimum data unit of census and land-use is unified into block.
	\item Matching the time points of data. To match the time point in different data sources, this study investigates the relationship between transit ridership and land-use using the data of 2008, for which is included in all kinds of data.
	\item Extracting the data onto population and land-use within the catchment area of subway stations. This operation is conducted by using ArcGIS, as shown in Figure \ref{fig:chp3:DataExtraction}, an 800-meter circle area taking the subway station as the center is drawn at first, then the statistics within this area is extracted.
	\setlength{\parskip}{0.7\baselineskip} % 设置段间距
\end{enumerate}

\begin{figure}[htbp]
	\centering
	\includegraphics[width=\linewidth]{chapter3/DataExtraction}
	\caption{Extraction of data}
	\label{fig:chp3:DataExtraction}
\end{figure}

%
\section{Characteristics of transit ridership and land-use}
%
\subsection{Characteristics of annual change in transit ridership}
%
The subway line 3 was put into operation from 2004, the data onto transit ridership was fully recorded from the year of 2005. Figure \ref{fig:chp3:RidershipVariation} gives the trend of transit ridership during 2005-2014. The total transit ridership has exceeded 0.6 million per day, of which the transit ridership of line 1 accounts for the most reaching about 0.5 million per day.

\begin{figure}[htbp]
	\centering
	\includegraphics[width=\linewidth]{chapter3/RidershipVariation}
	\caption{Variation in the transit ridership of Fukuoka subway 2005-2014}
	\label{fig:chp3:RidershipVariation}
\end{figure}

%
From the annual change rate of transit ridership in terms of subway lines, as shown in Figure \ref{fig:chp3:AnnualGrowth}, notably, at the first three years after the line 3 was opened, the growth rate of line 3 is much more than that of line 1 and line 2. After the year of 2009, the growth rates of three lines tend to be stable and consistent with each other.

%
\begin{figure}[htbp]
	\centering
	\includegraphics[width=\linewidth]{chapter3/AnnualGrowth}
	\caption{Annual growth rate of transit ridership by subway lines}
	\label{fig:chp3:AnnualGrowth}
\end{figure}

%
The characteristics of transit ridership are investigated from two aspects, the growth rate variation and transit ridership variation. Different types of station have different characteristics, specific to each subway station, the transit ridership and growth rate is shown in Figure \ref{fig:chp3:AnnualVariationEachStation}. It is sorted by ascending order for transit ridership. The transit ridership is the daily average on the year of 2010, the growth rate is the average of 10 years from 2005 to 2014. 

%
\begin{figure}[htbp]
	\centering
	\includegraphics[width=\linewidth]{chapter3/AnnualVariationEachStation}
	\caption{Variation in transit ridership of each station}
	\label{fig:chp3:AnnualVariationEachStation}
\end{figure}

%
Figure \ref{fig:chp3:RidershipSpatialDistribution} plotted the information of each station on the map, the stations in the newly opened line 3 generally have lower transit ridership but higher growth rate. The stations with the higher growth rate in line 1 and line 2 are mainly located close to the downtown area. From the spatial distribution of both transit ridership and growth rate, it shows a clear central agglomeration effect, which means the large-scale stations can attract more passengers.

%
\begin{figure}[htbp]
	\centering
	\includegraphics[width=\linewidth]{chapter3/RidershipSpatialDistribution}
	\caption{Spatial distribution of transit ridership}
	\label{fig:chp3:RidershipSpatialDistribution}
\end{figure}

%
\subsection{Analysis on the factors of land-use around subway stations}
%
\subsubsection{Data extraction}
%
As stated before, the data onto land-use is extracted from the 800-meter circle region taking subway station as the center. As shown in the Figure \ref{fig:chp3:BufferAreas}, 35 buffer areas with a 800 meters radius are created. Drawing the buffer area is conducted using the tool of “Buffer” in ArcGIS, extracting data is using the tool of “Tabulate Intersection”.

%
\begin{figure}[htbp]
	\centering
	\includegraphics[width=\linewidth]{chapter3/BufferAreas}
	\caption{800-meter Circular buffer areas for 35 subway stations}
	\label{fig:chp3:BufferAreas}
\end{figure}

%
The land-use data onto Urban Planning Basic Survey is available per 5 years, to match the data period of transit ridership, the land-use data in the year of 2008 is used for analyzing characteristics of land-use. The dataset of land-use contains 23 types, which, however, are too many for analyzing the characteristics of land-use in the case of only 35 stations. Moreover, most types of land-use account for very little of the building area, for which they are thought to have restricted influence on transit ridership. As a result, 9 types of land-use, which have relatively larger building areas and commonly exist in all the catchment area of subway stations, are selected to analyze the characteristics of land-use. The items are listed in Figure \ref{fig:chp3:IndicatorSelection}.

%
\begin{figure}[htbp]
	\centering
	\includegraphics[width=\linewidth]{chapter3/IndicatorSelection}
	\caption{Flow of selecting indicators}
	\label{fig:chp3:IndicatorSelection}
\end{figure}

%
Through the initial screening, there are 10 indicators being selected. Nevertheless, restricted by the sample size of 35 stations, the 10 indicators are still too many to quantitatively examine the characteristics of land-use. To explore what these indicators are expressing, the correlation analysis is put forward to make clear about the relationship between indicators. The result of correlation analysis is shown in the Table \ref{tab:chp3:CorrelationAnalysis}. There are strong correlations between the 4 types of land-use which are office, commerce, hotel, and entertainment. The type of dwelling upon shop also has a strong correlation between that of office, commerce, and hotel, while the culture type of land-use closely relates to the type of apartment and dwelling upon shop.

%
\begin{sidewaystable}[htbp]
	\centering
	\caption{Correlation analysis for indicators}
	\label{tab:chp3:CorrelationAnalysis}
	\small
	\renewcommand{\arraystretch}{1.5} % 重设表间距
	\begin{tabular}{p{9em}|p{3em}<{\raggedleft}|p{3em}<{\raggedleft}|p{3em}<{\raggedleft}|p{3em}<{\raggedleft}|p{3em}<{\raggedleft}|p{3em}<{\raggedleft}|p{3em}<{\raggedleft}|p{3em}<{\raggedleft}|p{3em}<{\raggedleft}|p{3em}<{\raggedleft}}
		\Xhline{1.5pt}
		Indicator & 1 & 2 & 3 & 4 & 5 & 6 & 7 & 8 & 9 & 10 \\
		
		\Xhline{0.5pt}
		1. Business & 1.000 & \multicolumn{1}{r|}{\cellcolor[rgb]{ 0.8, 0.8, 0.8} 0.848} & \multicolumn{1}{r|}{\cellcolor[rgb]{ 0.8, 0.8, 0.8} 0.987} & \multicolumn{1}{r|}{\cellcolor[rgb]{ 0.8, 0.8, 0.8} 0.775} & -0.385 & 0.378  & \multicolumn{1}{r|}{\cellcolor[rgb]{ 0.8, 0.8, 0.8} 0.848} & 0.615  & 0.006  & 0.656 \\
		
		2. Commerce & \multicolumn{1}{r|}{\cellcolor[rgb]{ 0.8, 0.8, 0.8} 0.848} & 1.000 & {\cellcolor[rgb]{ 0.8, 0.8, 0.8} 0.828}  & \multicolumn{1}{r|}{\cellcolor[rgb]{ 0.8, 0.8, 0.8} 0.765} & -0.293 & 0.345  & \multicolumn{1}{r|}{\cellcolor[rgb]{ 0.8, 0.8, 0.8} 0.734} & 0.645  & 0.065  & 0.530 \\
		
		3. Hotel & \multicolumn{1}{r|}{\cellcolor[rgb]{ 0.8, 0.8, 0.8} 0.987} & \multicolumn{1}{r|}{\cellcolor[rgb]{ 0.8, 0.8, 0.8} 0.828} & 1.000 & \multicolumn{1}{r|}{\cellcolor[rgb]{ 0.8, 0.8, 0.8} 0.716} & -0.361 & 0.382  & \multicolumn{1}{r|}{\cellcolor[rgb]{ 0.8, 0.8, 0.8} 0.815} & 0.585  & 0.000  & 0.619 \\
		
		4. Entertainment & \multicolumn{1}{r|}{\cellcolor[rgb]{ 0.8, 0.8, 0.8} 0.775} & \multicolumn{1}{r|}{\cellcolor[rgb]{ 0.8, 0.8, 0.8} 0.765} & \multicolumn{1}{r|}{\cellcolor[rgb]{ 0.8, 0.8, 0.8} 0.716} & 1.000 & -0.322 & 0.281  & 0.663  & 0.439  & -0.018 & 0.552 \\
		
		5. Residence & -0.385 & -0.293 & -0.361 & -0.322 & 1.000 & 0.205  & -0.287 & -0.164 & -0.175 & -0.063 \\
		
		6. Apartment house & 0.378  & 0.345  & 0.382  & 0.281  & 0.205  & 1.000 & 0.649  & 0.408  &0.087  & \multicolumn{1}{r}{\cellcolor[rgb]{ 0.8, 0.8, 0.8} 0.729} \\
		
		7. Dwelling with shop & \multicolumn{1}{r|}{\cellcolor[rgb]{ 0.8, 0.8, 0.8} 0.848} & \multicolumn{1}{r|}{\cellcolor[rgb]{ 0.8, 0.8, 0.8} 0.734} & \multicolumn{1}{r|}{\cellcolor[rgb]{ 0.8, 0.8, 0.8} 0.815} & 0.663  & -0.287 & 0.649  & 1.000 & 0.641  & 0.038  & \multicolumn{1}{r}{\cellcolor[rgb]{ 0.8, 0.8, 0.8} 0.808} \\
		
		8. Government & 0.615  & 0.645  & 0.585  & 0.439  & -0.164 & 0.408  & 0.641  & 1.000 & 0.395  & 0.549 \\
		
		9. Education & 0.006  & 0.065  & 0.000  & -0.018 & -0.175 & 0.087  & 0.038  & 0.395  & 1.000 & -0.012 \\
		
		10. Culture & 0.656  & 0.530  & 0.619  & 0.552  & -.063 & \multicolumn{1}{r|}{\cellcolor[rgb]{ 0.8, 0.8, 0.8} 0.729} & \multicolumn{1}{r|}{\cellcolor[rgb]{ 0.8, 0.8, 0.8} 0.808} & 0.549  & -0.012 & 1.000 \\
		\Xhline{1.5pt}
	\end{tabular}%
\end{sidewaystable}%

\subsubsection{Factor analysis}
%
To deal with the strong collinearity among indicators, also to make clear about the internal relationship, the exploratory factor analysis is introduced to reduce the dimension of the indicators, thus to explore the meanings of the indicators. The factor analysis is a statistical method used to describe variability among observed, correlated variables in terms of a potentially lower number of unobserved variables, for which the factor analysis is usually used for dealing with the indicator set with strong collinearity. The procedure of factor analysis is as below.

%
\emph{1. KMO and Bartlett's Test}

%
The KMO measure of sampling adequacy reflects the correlation between the indicators, referring to Table \ref{tab:chp3:KMOBartlettTest}. The test result of 0.756, which is greater than the suggested value of 0.7, means this indicator set has enough collinearity for conducting the factor analysis. As to the Bartlett’s Test, if the variables are independent from each other, the common factor cannot be extracted from it, and factor analysis cannot be applied. The Bartlett sphere test judges that if the correlation matrix is a unit matrix, the factor analysis method of each variable is invalid. The test results show a $Sig.<0.05$. It means each variable has correlation and factor analysis is effective. 

% Table
\begin{table}[htbp]
	\centering
	\caption{KMO and Bartlett's Test}
	\label{tab:chp3:KMOBartlettTest}
	\small
	\renewcommand{\arraystretch}{1.25} % 重设表间距
	\begin{tabular}{p{16em}<{\centering}rr}
		\Xhline{1.5pt}
		\multicolumn{2}{c}{Kaiser-Meyer-Olkin Measure of Sampling Adequacy} & 0.756 \\
		\midrule
		
		\multirow{3}[0]{16em}{\centering{Bartlett's Test of Sphericity}} & \multicolumn{1}{r}{Chi-Square} & 346.086 \\
		& df & 45 \\
		& Sig. & 0.000 \\
		\Xhline{1.5pt}
	\end{tabular}%
\end{table}%

%
\emph{2. Factor extraction}

%
The factors are extracted using principal component method, Table \ref{tab:chp3:Communalities} lists the communalities of the indicators. As shown in this table, most information contained in the indicators can be explained by the extracted factors. Less information loss during the process of extracting factors means a good effect.

%
% Table generated by Excel2LaTeX from sheet 'Sheet2'
\begin{table}[htbp]
	\centering
	\caption{Communalities of each indicator}
	\label{tab:chp3:Communalities}
	\small
	\renewcommand{\arraystretch}{1.25} % 重设表间距
	\begin{tabular}{lrr}
		\Xhline{1.5pt}
		Indicator & Initial & Extraction \\
		\midrule
		
		Office & 1.000 & 0.943 \\
		Commerce & 1.000 & 0.799 \\
		Hotel & 1.000 & 0.890 \\
		Entertainment & 1.000 & 0.725 \\
		Residence & 1.000 & 0.710 \\
		Apartment house & 1.000 & 0.849 \\
		Dwelling with shop & 1.000 & 0.885 \\
		Government & 1.000 & 0.756 \\
		Education & 1.000 & 0.923 \\
		Culture & 1.000 & 0.811 \\
		\Xhline{1.5pt}
	\end{tabular}
\end{table}%

%
As a result, four factors whose eigenvalue are greater than 1.00 are extracted. These four factors account for 82.90\% of all the variance. It means the three factors can explain most part of the original 10 indicators. The Table \ref{tab:chp3:TotalVarianceExplained} shows the total variance explained.

%
% Table generated by Excel2LaTeX from sheet 'Sheet2'
\begin{table}[htbp]
	\centering
	\caption{Total variance explained}
	\label{tab:chp3:TotalVarianceExplained}%
	\small
	\renewcommand{\arraystretch}{1.25} % 重设表间距
	\begin{tabular}{cccc}
		\Xhline{1.5pt}
		\multirow{2}[4]{*}{Component} & \multicolumn{3}{p{15em}}{Rotation Sums of Squared Loadings} \\
		\cmidrule{2-4}
		& Total & \% of Variance & Cumulative \% \\
		\midrule
		
		1 & 5.068 & 50.676 & 50.676 \\
		2 & 1.851 & 18.510 & 69.187 \\
		3 & 1.372 & 13.716 & 82.902 \\
		\Xhline{1.5pt}
	\end{tabular}%
\end{table}%

%
\emph{3. Factor naming and interpretation}

%
The rotated component matrix (Table \ref{tab:chp3:RotatedComponent}) shows the attribution of each indicator on the newly extracted factors. The three factors can be named and explained as follows, they are Office \& commerce, mixed-residence, education.

%
\begin{itemize}
	\item \textbf{Factor 1: Office \& commerce} \\
	This factor represents the land-use mainly including office area, large commercial area, and some commercial supporting facilities.
	
	\item \textbf{Factor 2: Mixed residence} \\
	The indicators of apartment, residence, and culture mainly attribute to this factor, which can reflect the attribute of residence with supporting facilities.
	
	\item \textbf{Factor 3: Education} \\
	Except for the indicator of education, the indicator of government also attributes a large part of this factor. It means the land-use of education and government have a relatively strong correlation.
\end{itemize}

% Table
\begin{table}[htbp]
	\centering
	\caption{Rotated Component Matrix}
	\label{tab:chp3:RotatedComponent}%
	\small
	\renewcommand{\arraystretch}{1.25} % 重设表间距
	\begin{tabular}{lp{3em}<{\centering}p{3em}<{\centering}p{3em}<{\centering}}
		\Xhline{1.5pt}
		\multirow{2}[3]{*}{Indicator} & \multicolumn{3}{c}{Component} \\
		\cmidrule{2-4}
		& 1 & 2 & 3 \\
		\midrule
		
		Office & \cellcolor[rgb]{ 0.8,  0.8, 0.8} 0.962 & 0.122 & 0.060 \\
		Hotel & \cellcolor[rgb]{ 0.8,  0.8, 0.8} 0.934 & 0.123 & 0.050 \\
		Commerce & \cellcolor[rgb]{ 0.8,  0.8, 0.8} 0.878 & 0.105 & 0.131 \\
		Entertainment & \cellcolor[rgb]{ 0.8,  0.8, 0.8} 0.850 & 0.044 & -0.030 \\
		Dwelling with shop & \cellcolor[rgb]{ 0.8,  0.8, 0.8} 0.834 & 0.417 & 0.125 \\
		Apartment house & 0.301 & \cellcolor[rgb]{ 0.8,  0.8, 0.8} 0.860 & 0.134 \\
		Residence & -0.521 & \cellcolor[rgb]{ 0.8,  0.8, 0.8} 0.620 & -0.234 \\
		Education & -0.066 & -0.039 & \cellcolor[rgb]{ 0.8,  0.8, 0.8} 0.958 \\
		Culture & \cellcolor[rgb]{ 0.8,  0.8, 0.8} 0.627 & \cellcolor[rgb]{ 0.8,  0.8, 0.8} 0.644 & 0.057 \\
		Government & 0.570  & 0.305 & \cellcolor[rgb]{ 0.8,  0.8, 0.8} 0.582 \\
		\Xhline{1.5pt}
	\end{tabular}%
	\label{tab:addlabel}%
\end{table}%

%
\subsection{Classification of stations}
%
To explore the characteristics of land-use in terms of each station, the cluster analysis is used to classify all the 35 subway stations, and the characteristics are summarized respect to the classifications.

%
The indicators used for classifying the subway station are selected on the base of correlation analysis and factor analysis conducted before. Although the indicator of office and commerce have a strong collinearity, the definitions and usage of them are quite different. Therefore, both of the indicators are put into the cluster analysis. Also with the consideration of the internal correlations shown in Table \ref{tab:chp3:CorrelationAnalysis}, the indicators with relatively high independence are selected. At last, there are 4 indicators of land-use being selected for classification, they are office, commerce, residence, education, adding to one more indicator of population density are selected to describe the characteristic of density.

%
As to the procedure of cluster analysis, the cluster method is Ward Method. The measurement of interval uses the Squared Euclidean Distance. Then the data is standardized into 0.00-1.00 range. The 34 of all the 35 subway stations fall into 5 types, while the station 7 located in the Fukuoka airport differs from the others. The 5 types are: low-density residence (9 stations), high-density residence (9 stations), education (6 stations), downtown commerce (5 stations), and office (5 stations) respectively. The result of classification is in the Figure \ref{fig:chp3:StationClassification}, the right part of this Table is the pie chart of land-use which shows the proportion of land-use. The description of each type of stations is given below. 

%
\begin{figure}[htbp]
	\centering
	\includegraphics[width=\linewidth]{chapter3/StationClassification}
	\caption{Classification of stations in terms of land-use}
	\label{fig:chp3:StationClassification}
\end{figure}

%
\begin{itemize}
	\item \textbf{Characteristics of low-density residence type} \\
	This type has 9 stations with the characteristics of high proportion of residence, and low population-density, very few office and commerce.
	
	\item \textbf{Characteristics of High-density residence type} \\
	The proportion of each land-use in high-density residence is similar with that in low-density residence type, but population-density is much higher than other types.
	
	\item \textbf{Characteristics of education type} \\
	This type has 6 stations, they are located in the vicinity of colleges and universities, and the service population is mainly students and nearby residents.
	
	\item \textbf{Characteristics of downtown commerce type} \\
	The type of downtown commerce includes 5 stations, which constituted the CBD area of Fukuoka. These stations have the highest proportion of office and commerce, while proportion of residence is the lowest and population density is relatively lower.
	
	\item \textbf{Characteristics of office type} \\
	This type includes 5 stations, of which the main land-use is office. These stations mainly distribute over the CBD area, of which the mixed land-use is one of the main characteristics.
\end{itemize}

%
The station 7, which locates at Fukuoka airport, is different from the other 5 types of stations, the main component of land-use is transportation facilities. From the view of land-use, reflecting on the result of classification, the station 7 does not belong to any type of stations.

%
\section{Exploration on the relationship between land-use and transit ridership}
\subsection{General relationship between land-use and transit ridership}
%
The data onto land-use is updated per 5 years, while the data onto subway transit ridership is annual. The 2008 data included in both data is used to analyze the relationship between subway transit ridership and land-use.

%
According to the classification result (refer to \ref{fig:chp3:StationClassification}) and the average transit ridership of each type of station, which is shown in the Table \ref{tab:chp3:ClassificationCharacteristics}, the subway transit ridership varies a lot due to different types of land-use. The station of low-density residence type has an average transit ridership of only 3,416, while that of high-density residence type has 15,633 on average. The downtown commerce type has the highest average transit ridership of 52,300, while that in office type is only 11,038, even though similar with the type of downtown commerce the office type has a high proportion of land-use in office and commerce. The proportions of office and commerce in education type are similar to that in the two residence types, but from which the average transit ridership is quite different.

% Table
\begin{table}[htbp]
	\centering
	\caption{Characteristic of each classification}
	\label{tab:chp3:ClassificationCharacteristics}
	\small
	\renewcommand{\arraystretch}{1.25} % 重设表间距
	\begin{tabular}{lp{4em}<{\raggedleft}p{4em}<{\raggedleft}p{4em}<{\raggedleft}p{4em}<{\raggedleft}p{4em}<{\raggedleft}}
		\Xhline{1.5pt}
		Type & Office & Commerce & Residence & Education & Transit ridership \\
		\midrule
		
		Low-density residence & 2.85\% & 7.99\% & 69.23\% & 4.47\% & 3416 \\
		High-density residence & 5.43\% & 4.03\% & 59.66\% & 7.47\% & 15633 \\
		Education & 6.03\% & 5.08\% & 40.22\% & 20.71\% & 7391 \\
		Downtown commerce & 38.08\% & 25.76\% & 13.68\% & 1.24\% & 52300 \\
		Office & 27.48\% & 14.20\% & 27.09\% & 3.10\% & 11038 \\
		\Xhline{1.5pt}
	\end{tabular}
\end{table}

%
According to the classification result of both subway transit ridership and land-use around the stations, the cross table is shown in the Figure \ref{tab:chp3:CrossTable} below. All the stations belonging to low-density residence type have small-scale of transit ridership, which is under 10,000 per day. Both the two hub stations belonging to the downtown commerce type located in the CBD area of Fukuoka. Overall, it can be inferred from this table that the area with lower density either in population or building has a lower demand on using rail transit; besides the high density in population and building, hub stations should also have particularities in location and function. 

% Table
\begin{table}[htbp]
	\centering
	\caption{Cross tabulation of land-use and transit ridership}
	\label{tab:chp3:CrossTable}
	\small
	\renewcommand{\arraystretch}{1.25} % 重设表间距
	\begin{tabular}{lp{3em}<{\raggedleft}p{3em}<{\raggedleft}p{3em}<{\raggedleft}p{3em}<{\raggedleft}}
		\Xhline{1.5pt}
		Type & Hub & Large-scale & Medium-scale & Small-scale \\
		\midrule
		
		Low-density residence & 0 & 0 & 0 & 9 \\
		High-density residence & 0 & 2 & 4 & 3 \\
		Education & 0 & 0 & 1 & 5 \\
		Downtown commerce & 2 & 1 & 2 & 0 \\
		Office & 0 & 0 & 3 & 2 \\
		\Xhline{1.5pt}
	\end{tabular}
\end{table}

%
Through the characteristics analysis of subway transit ridership and land-use, the general trend of the relationship between them can be understood. However, the conclusion obtained from the trend analysis cannot be fully trusted due to the lack of quantitative analysis, also, how the factor land-use can influence transit ridership is still not clear.

\subsection{Statistical relationship of land-use and transit ridership}
%
The quantification method \uppercase\expandafter{\romannumeral1} is introduced to explore the influence that each factor impact on the transit ridership. Unlike ordinary regression models, the quantification method \uppercase\expandafter{\romannumeral1} divides the continuous independent variables into categorical variables. Then estimating the correlation between categorical independent variables and the continuous dependent variable. This method is thought suitable to do the exploratory analysis on the data onto the first time due to the procedure of discretizing the continuous independent variables. This discretization can partly reduce the deviation caused by the uneven distribution of the sample.

%
The quantification method \uppercase\expandafter{\romannumeral1} can be viewed as an improvement on ordinary regression models which is used for dealing with the exploratory analysis at the beginning. Therefore, the indicators having strong collinearity should be excluded in the process of selecting explaining indicators. Referring to the Table \ref{tab:chp3:CorrelationAnalysis}, there is no significant linear correlation between these indicators of office, residence, education and government. Notably, the indicators of office and commerce have strong collinearity, however, both the indicators account for a large part in total, for which they should not be easily ignored. In order to reserve more information, a new indicator named commerce \& office which represents the sum of commerce and office building area is proposed. Coupled with the indicator of population density, there are 5 influencing indicators selected as the independent variables. The division of continuous variables is determined based on the Squared Euclidean distance between groups. Both the transit ridership and growth rate of transit ridership are estimated using the 5 indicators of population density, commercial \& office, residence, education, and government. The result is shown as Table \ref{tab:chp3:QM1TransitRidership} and Table \ref{tab:chp3:QM1GrowthRate}.

%
As the results, the coefficients of determination with the value of 0.513 and 0.537 in both models respectively are not satisfactory. However, the results are not for predicting the transit ridership in the future but for exploring the influence of land-use on the transit ridership. In view of this, the results are thought to have a certain referential value. 

%
As to the result of influence on the transit ridership, the commerce \& office contributes most of the variation in transit ridership, which means the commerce \& office plays important role in explaining the transit ridership. The building area of education has the weakest influence, while even building area of government is much less than that of education, the indicator of government contributes more to the variation in transit ridership. 

% Table
\begin{table}[htbp]
	\centering
	\caption{Results of quantification method \uppercase\expandafter{\romannumeral1} on transit ridership}
	\label{tab:chp3:QM1TransitRidership}
	\small
	\renewcommand{\arraystretch}{1.25} % 重设表间距	
	\begin{tabular}{p{8em}rp{4em}<{\raggedleft}p{4em}<{\raggedleft}p{4em}<{\centering}}
		\Xhline{1.5pt}	
		Factor category & Category & Number & Score & \multirow{1}[1]{4em}{Range} \\
		\midrule
		
		\multirow{5}[0]{8em}{population density \\ (person/ha)} & 0-40  & 3 & 4644 & \multirow{5}[0]{4em}{13004 \\ 13.84\%} \\
		& 40-80 & 13 & 2283 & \\
		& 80-120 & 8 & 164 &  \\
		& 120-160 & 8 & -2481 &  \\
		& 160- & 3 & -8360 &  \\
		\midrule
		
		\multirow{4}[0]{8em}{Commerce \& Office \\ ($m^2$)} & 0-100,000 & 16 & -6249 & \multirow{4}[0]{4em}{43288 \\ 46.07\%} \\
		& 100,000-400,000 & 9 & -6617 &\\
		& 400,000-1,000,000 & 5 & -4765 & \\
		& 1,000,000- & 5 & 36671 & \\
		\midrule
		
		\multirow{3}[0]{8em}{Residence \\ ($m^2$)} & 0-300,000 & \multicolumn{1}{r}{4} & -1450 & \multirow{3}[0]{4em}{16240 \\ 17.28\%} \\
		& 300,000-800,000 & 20 & -5575 & \\
		& 800,000- & 11 & 10664 & \\
		\midrule
		
		\multirow{3}[0]{8em}{Government \\ ($m^2$)} & 0-1,000 & 13 & -2957 & \multirow{3}[0]{4em}{12267 \\ 13.06\%} \\
		& 1,000-10,000 & 9 & -5501 & \\
		& 10,000- & 13 & 6766 & \\
		\midrule
				
		\multirow{4}[0]{8em}{Education \\ ($m^2$)} & 0-10,000 & 5 & 4842 & \multirow{4}[0]{4em}{9159 \\ 9.75\%}\\
		& 10,000-50,000 & 11 & 993 & \\
		& 50,000-100,000 & 11 & -4317 & \\
		& 100,000- & 8 & 1544 & \\
		\Xhline{0.5pt}
				
		\multicolumn{2}{c|}{Independent variable} & \multicolumn{2}{c}{Sample size} & 35 \\
		\multicolumn{2}{c|}{Transit ridership} & \multicolumn{2}{c}{Coefficient of determination} & 0.513 \\
		\Xhline{1.5pt}
	\end{tabular}%
\end{table}%

% Table
\begin{table}[htbp]
	\centering
	\caption{Results of quantification method \uppercase\expandafter{\romannumeral1} on growth rate of transit ridership}
	\label{tab:chp3:QM1GrowthRate}
	\small
	\renewcommand{\arraystretch}{1.25} % 重设表间距	
	\begin{tabular}{p{8em}rp{4em}<{\raggedleft}p{4em}<{\raggedleft}p{4em}<{\centering}}
		\Xhline{1.5pt}	
		Factor category & Category & Number & Score & \multirow{1}[1]{4em}{Range} \\
		\midrule
		
		\multirow{5}[0]{8em}{population density \\ (person/ha)} & 0-40  & 3 & 0.0155 & \multirow{5}[0]{4em}{0.0265 \\ 28.44\%} \\
		& 40-80 & 13 & 0.0020 & \\
		& 80-120 & 8 & -0.0001 & \\
		& 120-160 & 8 & -0.0110 & \\
		& 160- & 3 & 0.0056 & \\
		\midrule
		
		\multirow{4}[0]{8em}{Commerce \& Office \\ ($m^2$)} & 0-100,000 & 16 & -0.0016 & \multirow{4}[0]{4em}{0.0097 \\ 10.44\%} \\
		& 100,000-400,000 & 9 & 0.0006 & \\
		& 400,000-1,000,000 & 5 & 0.0069 & \\
		& 1,000,000- & 5 & -0.0028 & \\
		\midrule
		
		\multirow{3}[0]{8em}{Residence \\ ($m^2$)} & 0-300,000 & \multicolumn{1}{r}{4} & -0.0183 & \multirow{3}[0]{4em}{0.0226 \\ 24.31\%} \\
		& 300,000-800,000 & 20 & 0.0043 & \\
		& 800,000- & 11 & -0.0012 & \\
		\midrule
		
		\multirow{3}[0]{8em}{Government \\ ($m^2$)} & 0-1,000 & 13 & 0.0111 & \multirow{3}[0]{4em}{0.0228 \\ 24.54\%} \\
		& 1,000-10,000 & 9 & 0.0008 & \\
		& 10,000- & 13 & -0.0117 & \\
		\midrule
		
		\multirow{4}[0]{8em}{Education \\ ($m^2$)} & 0-10,000 & 5 & -0.0078 & \multirow{4}[0]{4em}{0.0114 \\ 12.27\%}\\
		& 10,000-50,000 & 11 & -0.0025 & \\
		& 50,000-100,000 & 11 & 0.0034 & \\
		& 100,000- & 8 & 0.0036 & \\
		\Xhline{0.5pt}
		
		\multicolumn{2}{c|}{Independent variable} & \multicolumn{2}{c}{Sample size} & 35 \\
		\multicolumn{2}{c|}{Growth rate of transit ridership} & \multicolumn{2}{c}{Coefficient of determination} & 0.537 \\
		\Xhline{1.5pt}
	\end{tabular}
\end{table}


%
The same indicator set is also used for estimating the influence on the growth rate of transit ridership (Table \ref{tab:chp3:QM1GrowthRate}), as a result, the coefficient of determination is a little higher than that of transit ridership. The indicator of commerce \& office explains only 10\% of the total variation in growth rate, while it accounts for almost 50\% in the case of transit ridership. This result shows that the driving force of transit ridership and of transit ridership growth rate is different. The factor of population and residents play the key roles in promoting the use of subway.

%
\section{Conclusion}
%
This study investigated the variation of subway transit passengers from 2005 to 2014, and then analyzed the types of land-use around the stations. On the base of understanding the characteristics of transit ridership and land-use, the relationship between them was also estimated. The 35 subway stations were classified into 5 types with typical characteristics in terms of land-use. The transit ridership of each type of stations showed significant differences. The results from quantification method \uppercase\expandafter{\romannumeral1} showed the quantitative relationship between transit ridership and land-use. Even though the accuracy of results was not enough to make a prediction, it provided references for selecting more valid indicator to make a prediction in the future research.

%
The major finding of this study can be summarized as follows. 

%
\begin{itemize}
	\item From the comprehensive description of the study case and the investigation into the transit ridership, it can be known that the subway transit ridership is still increasing to date, but it probably turns to decrease in the near future. 
	
	\item The subway line 3 has a greater potential for growth in transit ridership. Even though the transit ridership also the population and building density are still lower at present, the stations in subway line 3 are under rapid developing.
	
	\item The spatial variation in transit ridership shows the characteristics of central aggregation. The hub stations with higher transit ridership near to the downtown area have a higher growth rate in transit ridership.
	
	\item According to the classification of stations in terms of land-use, different types of stations have quite different scales on the transit ridership.
	
	\item The same indicators of land-use have different effects on transit ridership and the growth rate of transit ridership.
\end{itemize}

%
This study is the first step to explain the influencing factors of rail transit ridership, which aims to give a comprehensive description of the research objects also provide references to further explaining transit ridership. Based on the understanding of the insufficiencies in this study, some recommendations are given for the next research.

%
\begin{itemize}
	\item The determination of catchment area needs more investigation. Since the 800-meter circle catchment does not consider the differences in the form of the road network, the same 800 meters catchment area may represent different walking distance reflecting on the real road network.
	
	\item The selection of indicators explaining the transit ridership needs more exploration. The other categories of indicators about such as facilities, socio-economic, and urban design should also have influence on the transit ridership.
	
	\item The selection and usage of estimation models need more investigation. The issue of transit ridership is not only a simple regression problem, but it also relates to the location of stations. A model which is suitable for spatial analysis may be better than an ordinary regression model. 
	
	\item The approach to tackle the small sample case should be considered. Statistical analysis needs a certain sample size, otherwise, it cannot say the estimation result is credible. The problem led by small sample also reflects on the procedure of selecting the valid explanatory variables. 
\end{itemize}


% reference
\clearpage % 新起一页
\bibliographystyle{apacite}
% \bibliographystyle{IEEEtran}
\bibliography{ref}